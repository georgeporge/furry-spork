\documentclass[..\Papers.tex]{subfiles}

\begin{document}

\section{Yeager2015}
\citep{Yeager2015}


\begin{itemize}
    \item "They pointed out that JEBAR arises because topography causes the PV contours at difference depths (densities) to diverge, and so circula- tion that is essentially baroclinic (with PV-conserving flow in each layer) can acquire a barotropic component (which does not follow barotropic PV contours)"
    \item "Zhang and Vallis (2007) have shown that the BPT associated with deep western boundary current (DWBC) flow offshore of the Grand Banks is a key factor in setting the strengthof the northern recirculation gyre (NRG)�the cyclonic, barotropic flow that has been observed between the northern flank of the GSand the GrandBanks (Hogg et al. 1986). They showed that BPT-related changes in the NRG can influence the GS path after separation from Cape Hatteras."
    \item "The relationship between JEBAR and BPT has been clarified by, among others, Mertz and Wright (1992), Greatbatch et al. (1991), and Bell (1999); the former arises in the (potential) vorticity equation of the vertically averaged horizontal flow, whereas the latter arises in the vorticity equation of the vertically integrated horizontal flow"
    \item "JEBAR represents the component of BPT associated with the baroclinic (buoyancy de- pendent) part of the pressure gradient, and therefore it vanishes in the absence of stratification"
    \item "BPT can be nonzero regardless of stratification because it represents the projection of hori- zontal geostrophic bottom flow normal to isobaths. With the condition of no normal flowat the ocean bottom,BPT can be understood as a geostrophic bottom vortex stretching "

\end{itemize}

\paragraph{Appendix A has good equations?}


\end{document}
