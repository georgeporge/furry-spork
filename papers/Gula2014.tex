\documentclass[..\Papers.tex]{subfiles}

\begin{document}

\section{Gula2014}
\citep{Gula2014}


\begin{itemize}
    \item Good description of GS : passes through Strait of Florida then flows northward pressed against the confining wall of the southeastern U.S. continental shelf belfore leaving the slope at Cape Hatteras.
    \item GS path controlled by combination of boundary shape, bottom topography, entrainment of fluid from the gyre interior, and the adjustment of the flow tothe incrase in planetary vorticity as fluid is advected northward.
    \item cyclonic eddies propagate along the shelf (on the inshore side of the gulf Stream) - these are "frontal eddies" and occur where the Gulf Stream interacts with the slope and shelf. They're formed of deep, upwelled, cold domes.
    \item Eddies and meanders have strong implications for the biological production in the South Atlantic Bight (cites Lee et al. 1991) - important for biological reasons! (This is often one of the ways you can identify the Gulf Stream from Satellite images - e.g. ocean circulation book cover I think \todo{check this}).
    \item Charleston bump disrupts the Gulf Stream, deflecting it further Eastward and generating large meanders. The pressure forces at the bump generate drags and torques likely to impact strongly the momentum and corticity balances of the stream (which remain to be quantified).
    \item Charleston bump - preferred region for eddy generation using satellite-based measurements and statistics.
    \item other papers (Hood and Bane 1983 \& Dewar and Bane 1985) show both a large mean-to-eddy conversion at the bump and eddy-to-mean conversion downstream.
    \item mean-to-eddy conversion where the mean velocity gradient is the source for eddy generation through instability processes. Implies a down-gradient momentum transport and a deceleration of the mean flow.
    \item eddy-to-mean conversion more counterintuitive as it requires an upgradient momentum transport. The eddies are accelerating the mean flow.
    \item Xie et al 2007 conclude that the isobathic curvature plays a role in enhancing the baroclinic and barotropic energy conversion, whereas the bump provides a local mechanism to maximize the energy transfer rate. 
\end{itemize}

\paragraph{Explanation of ???}


\end{document}
