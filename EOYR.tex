\documentclass[a4paper,11pt]{article}

% With the commands after % signs you can define your own page size. Remove % to activate them and insert the values you want to define the size of the page you want.
    %\voffset=-2cm
    %\hoffset=-0.5cm
    %\textwidth=15cm
    %\parindent 0pt
    %\parskip 2ex
\setlength{\parindent}{24pt}
\setlength{\oddsidemargin}{-5mm}
\setlength{\evensidemargin}{-5mm}
\setlength{\textwidth}{165mm}
\setlength{\textheight}{230mm}
\setlength{\topmargin}{-10mm}
\setlength{\marginparwidth}{15mm}
\setlength{\marginparsep}{7mm}
%\setlength{\cftsecnumwidth}{2.3cm}


\usepackage{datetime}
\usepackage{graphicx,natbib}


\usepackage[colorinlistoftodos]{todonotes}

\usepackage{titling}
\newcommand{\subtitle}[1]{%
  \posttitle{%
    \par\end{center}
    \begin{center}\large#1\end{center}
    \vskip0.5em}%
}

%provide smart links in your document
\usepackage{hyperref}

% use double spacing for easy markup on the document for corrections
%\usepackage{setspace}
%\doublespacing


\newcommand{\tick}{\ding{51}}
\newcommand{\cross}{\ding{55}}

\newcommand\T{\rule{0pt}{3.5ex}}       % Top strut
\newcommand\B{\rule[-2.3ex]{0pt}{0pt}} % Bottom strut

\newdateformat{mydate}{%
    \monthname[\THEMONTH] \THEYEAR%
    }

\begin{document}
\bibliographystyle{spr-mp-sola}

% Front matter 
\title{End of First Year Report}
\subtitle{Draft}
\author{Georgina Long}

\date{\mydate\today}

\maketitle


\section{Introduction}

Ocean eddies are the subject of many ongoing investigations into ocean dynamics. The topic comprises a broad range of smaller-scale dynamics and relationships at varying scales. From energy transfer to blah blah *citations*, researchers are still trying to get a better understanding of the internal dynamics of the ocean. *citations*. Our main focus will be on the area of the Gulf Stream and mainly the interaction and effect of the bathymetry on the separation and subsequent path of the Gulf Stream.
\todo{Haven't added citations}

\subsection{The impact of an accurately resolved Gulf Stream}
\begin{itemize}
  \item Impact on other weather events - e.g.  \citep{Scaife2011a}
  \item Effect on other ocean processes and circulation - "blue spot of death"    %\todo{NW corner. Forntal path to west}
\end{itemize}
The Gulf Stream is part of the AMOC and is essential to transferring heat from the Gulf of Mexico at lower latitudes towards western and northern Europe. Fundamentally it gives north-western Europe its mild climate and thus has large impacts on the weather and climate as the heat is transferred into the atmosphere. This can have wide ranging effects on weather prediction – notably winter blocking \citep{Scaife2011a} and *other citation* \todo{I'm sure I read about cyclones somewhere - find citation}. Many lower resolution models render an unrealistic Gulf Stream, with the separation from the U.S. coast too far to the South, causing an inaccurate path from this point onwards. This obviously has a knock-on effect as the corresponding sea surface temperatures (SSTs) are much higher/lower than observed in the surrounding areas \citep{Greatbatch2004}\todo{check this is the right citation}. The cold bias in the North West corner of the Atlantic resulting from a poorly placed Gulf Stream has become known as the "blue spot of death" \citep{Gnanadesikan2007} to modellers due to the inaccuracies. The Northern Recirculation Gyre (NRG) interacts with the Gulf Stream and it has been noted by \citep{Zhang2007} and \citep{Ezer2016b}\todo{check this citation} that the representation of the Gulf Stream can have large effects on the energy modelled within the NRG \todo{\& the subtropical gyre too? – check this}. A lethargic NRG subsequently affects the *citations/TODO find out about this* and hence we can see that a poorly resolved Gulf Stream can have much wider implications throughout the model.

\subsection{Problems and Restrictions}
\begin{itemize}
  \item Many people restricted to coarser resolution models due to the cost of high resolution models but need high resolution to get a more accurate Gulf Stream - it's expensive to be accurate
  \item We don't understand the smaller processes involved. \citep{Nikurashin2012a}
\end{itemize}
Higher resolution models have been shown to provide a more realistic Gulf Stream path *citations*, however this comes at a cost requiring more processing power and thus a higher price for the model. It is generally understood *citations \citep{Nikurashin2012a} etc.* that this improvement is due to the higher resolution models being able to resolve small-scale processes which are lost in the coarser resolution counterparts. \citep{Ezer2016b} suggests that to replicate an accurate Gulf Stream separation, the model not only needed to resolve the Gulf stream but also the northern branches of NRG, the southward slope and shelf currents. \todo{Could the bathymetry cause an unresolved northern branches of NRG?}. *citation* It is also thought that the lack of detailed representation of the bathymetry in key areas play a large role in the unrealistic Gulf Stream path. *citation* posed that it is the interaction with the bathymetry and the consequential small-scale processes which cause the Gulf Stream to split off from the coast at Cape Hatteras and direct it’s path from there. The bathymetry itself can be represented in many different ways and the vertical coordinate system chosen has a large impact on the interaction of the ocean dynamics and the ocean floor. \citep{Ezer2016b} revisited a discussion on different vertical coordinate systems and their effect on the path of the Gulf Stream, though noted that the tested z-coordinate system did not include any partial or shaved cells. \todo{partial \& shaved cells - Mike's paper?}


\section{Background}
Tell a story up to where we are and what's missing.
\subsection{Literature Review (take most from coursework).Cover the various angles including:}
\begin{itemize}
  \item Modelling \& vertical grid systems - include partial cells
  \item Turbulence and bathymetry - understanding the smaller processes (vorticity etc.) \citep{Tansley2001} \citep{Nikurashin2012a}
  \item Possible relationships/data sets to see the history of the Gulf Stream. (Palaeo?) \citep{Ezer2015}
  \item Inclulde some phenomenology - e.g. bottom vortex stretching and bottom pressure torque. Maybe also the weird sandwhich diagram!
\end{itemize}


\section{Results}

Results so far - link into Background \& lead nicely into the next section.
\subsection{Barotropic vorticity diagnostics}
\begin{itemize}
  \item Results so far
  \item Discussion on results and possible implications
  \item Results on what we're looking at and why
  \item Maybe brief description of JEBAR etc. to explain it's use \& who else has used it to show what - e.g. why are we looking at it etc?
  \item Should probably include model set up
\end{itemize}



\section{Future Aims/Questions}

We seek a robust \& accurate Gulf Stream path. How can we improve the representation \& thus understanding of the Gulf Stream?

We seek to improve the understanding and representation of the Gulf Stream in moderate resolution models. As we have discussed, there are many processes involved in the dynamics of the ocean and thus there are many different directions to turn to when seeking solutions to this problem. Here we will focus on the interactions and effects of the bathymetry while being careful not to ignore other paths this investigation may lead down. 

\subsection{Identifying key processes}
\begin{itemize}
  \item Understanding
  \item Representation
\end{itemize}

It is speculated *paper*\todo{female author?} that interaction with the bathymetry creates small-scale turbulence and instabilities which can cause bathymetric steering and divert currents. If this is not being represented in coarser models, the energy behind the turbulence must be going elsewhere. \todo{paper}*paper* compared simulated kinetic energy over different models against measurements taken and noted that in some areas the total kinetic energy was being held higher in the ocean in the models than observed. It is discrepancies \todo{choose a different word} such as this which could have much wider implications. If the energy is not penetrating to the ocean floor, we cannot expect to be able to replicate the effects of the bathymetry. Perhaps pulling this energy further down (closer to observed values), we would be able to see the energy transfer from \todo{?}*** to the turbulence resulting from the bathymetry. 

Of course as previously discussed there are many processes which could contribute to a Gulf Stream path staying too far South. \citep{Ezer2016b} speculated that amongst other things, the northern branches of the NRG would have to be resolved in order to produce an accurate Gulf Stream. \citep{Zhang2007} determined that a significant contribution to the generation of the NRG is the bottom vortex stretching resulting from a downslope DWBC, which is in turn the result of the interaction with the bathymetry as the DWBC crosses the path of the Gulf Stream. Thus we see that this circles back to interaction with the bathymetry.


\subsection{Different model formulations}
\begin{itemize}
  \item Representation of the bathymetry
  \item Effects of different vertical grids
\end{itemize}

The many different ocean and OGCM models available allow us to see the effects of many different schemes and parameterisations in use today. In particular the choice of vertical coordinate system has long been discussed, especially in regard to its relevance here. Obviously the representation of the topography and the way it is rendered in the model is a result of the vertical coordinate choice. \citep{Ezer2016b} revisited the discussion in relation to the Gulf Stream separation and used similar models to determine the most accurate results over different choices for coordinate systems. However, the z-coordinate system chosen did not include any partial or shaved cells. It is inherently obvious that using the z-coordinate system would cause the various slopes in the ocean floor to appear as ‘steps’ which would significantly alter the flow around the area. \todo{A diagram of the different coordinate systems?}. It is not so surprising perhaps that under these circumstances \citep{Ezer2016b} found sigma or s-coordinates to be the most realistic given that they allow for a smooth ocean floor. It is perhaps natural then to question the effect of partial or shaved cells on these findings. As NEMO allows for such a diverse range of configurations and schemes, it would be interesting to recreate these results while implementing a z-coordinate system with partial cells to allow for a new take on this review. \todo{Maybe move to the next section instead?}

\subsection{Gulf Stream interaction with other systems}
\begin{itemize}
  \item Interaction with other currents   (DWBC, etc.)
  \item Relationship with data sets
  \item Evolution of the Gulf Stream
\end{itemize}



\section{Lines of Enquiry \& Methodology}

\subsection{NEMO}
\begin{itemize}
  \item Model Resolution \citep{Ezer2016b}
  \begin{itemize}
    \item simple configurations? \citep{Tansley2001}
  \end{itemize}
  \item Representation of Bathymetry
  \begin{itemize}
    \item Vertical Coordinate Systems
    \item Partial Cells
    \item Nested grids (enhancing resolution in specific areas?) 
  \end{itemize}
\end{itemize}

To examine the aforementioned interplay between the NRG and the path of the Gulf Stream, a simple configuration of NEMO simulating the double gyre set-up seen by the NRG north of the Gulf Stream and the subropical(?) gyre south of the gulf stream could be used to examine the effects in an idealised model. Results from this could help to influence our investigations within the more realistic models. As \citep{Tansley2001} used the classic problem of flow past a cylinder to understand the Gulf Stream separation, we too can learn from idealised set up and by breaking the problem down to its fundamental aspects.

\subsection{Understanding the Gulf Stream}
\begin{itemize}
  \item Turbulence    (\&Geostrophic turbulence)
  \item Vorticity and it's contributors
\end{itemize}
  
It has been discussed previously \todo{here? Or do I need to recite?} that a lack of energy transfer, either by lack of interaction with the bathymetry, or via some other process, could be to blame for the low levels of turbulence found in the area surrounding the separation of the Gulf Stream. As seen when \citep{Tansley2001} used a larger Reynolds number, allowing for more turbulent flow, flow passing a quarter-cylinder (akin to the coastline at Cape Hatteras), formed jet like streams with smaller eddies breaking away  from it. This is akin to satellite observations of the Gulf Stream \todo{could I put a side by side of the two pictures here? Satellite v Tansley/Marshall picture?}. This leads to the question which many try to answer – how can we improve/enhance the turbulence in a model?   
  
\subsection{Gulf Stream Relationships}
\begin{itemize}
  \item Data Sets    \citep{Ezer2015}
  \begin{itemize}
    \item Explore the possible links between the Gulf Stream and other data sets (e.g. coastal sea level)
  \end{itemize}
  \item Impact of other currents    \citep{Ezer2015}
  \item Changes in the Gulf Stream \citep{Greatbatch1991} \citep{Ezer2015}
\end{itemize}


\section*{Notes}
\begin{itemize}
	\item Make lit review more specific - focus it on a few key papers rather than covering a ton of them!
	\item Focus on phenomenology rather than model stuff
	\item Add a section somewhere to explain bottom pressure torque and bottom vortex stretching - maybe in diagnostics secion but elsewhere too
	\item Maybe add in weird sandwhich scales phenomenology?
	\item Explain vertically integrated blah blah
	\item Find out a use for the vertical average info.

%\bibliography{EoFYR}
\bibliography{../../../../bibtex/EoFYR}

%\newpage
%
%\appendix
%
%\part*{Appendices}
%
%
%\begin{thebibliography}{99}
%\bibitem{peccei}
%
%\bibitem{feynmandiag}
%
%\end{thebibliography}

\end{document}
