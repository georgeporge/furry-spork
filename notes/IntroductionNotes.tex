\documentclass[..\EOYR.tex]{subfiles}

\begin{document}

\section{Introduction}

\todo{Include Gulf Stream dimensions - or maybe mention if talking about scaling at some point?}
Ocean eddies are the subject of many ongoing investigations into ocean dynamics. The topic comprises a broad range of smaller-scale dynamics and relationships at varying scales. From energy transfer to blah blah *citations*, researchers are still trying to get a better understanding of the internal dynamics of the ocean. *citations*. Our main focus will be on the area of the Gulf Stream and mainly the interaction and effect of the bathymetry on the separation and subsequent path of the Gulf Stream.
\todo{Haven't added citations}

\paragraph{The impact of an accurately resolved Gulf Stream}
\ignore{ %TODO LIST
    \begin{itemize}
      \item Impact on other weather events - e.g.  \citep{Scaife2011a}
      \item Effect on other ocean processes and circulation - "blue spot of death"    \todo{NW corner. Forntal path to west}
      \item Global warming etc? - \textcolor{magenta}{resolution important as long integrations not affordable at high resolution.}
    \end{itemize}}
The Gulf Stream is part of the AMOC and is essential to transferring heat from the Gulf of Mexico at lower latitudes towards western and northern Europe. Fundamentally it gives north-western Europe its mild climate and thus has large impacts on the weather as the heat is transferred into the atmosphere.
This can have wide ranging effects on weather prediction such as winter blocking \citep{Scaife2011a} \td{*other citations and weather impacts*} \td{I'm sure I read about cyclones somewhere - find citation}. Variations in the Gulf Stream have also been linked to flooding on the Eastern U.S. coast caused by the increase in sea surface level by \citep{Ezer2015} among others. \\
It has been identified that the Gulf Stream has evolved over time, with \citep{Greatbatch1991} identifying a weakening of ~30Sv between data from 1955-59 and 1970-74. \td{Include about shutting down of GS in past.} The dependence on the Gulf Stream for weather gives greater importance to our understanding and ability to predict the changes in the Gulf Stream. \todo{Check this snentence}.
Many lower resolution models render an unrealistic Gulf Stream, \ignore{do I need to reference this?}with the separation from the U.S. coast too far to the South, causing an inaccurate path from this point onwards.
This causes knock-on effects as the corresponding sea surface temperatures (SSTs) are highter/lower than observed in the surrounding areas \citep{Greatbatch2004}\ignore{Maybe include diagram from Greatbatch to show discrepancies/differences}, and the continuing path of the Gulf Stream is innaccurate from the earlier misrepresentation.
Given the importance of climate change and research into global warming, accurate representations of SSTs are vital for realistic predictions of the planets climate.
Among the many processes at play are two gyres, the \td{subpolar} Norther Recirculation Gyre (NRG)\todo{Do I need caps for this?} which lies to the North of the Gulf Stream, and the \td{subtropical} \td{Worthington} gyre to the south of the Gulf Stream. These two gyres contribute to the transport of the Gulf Stream \citep{Hogg1986}, and can be seen to weaken, strengthen or be displaced in response to changing dynamics in many of the models used to investigate the Gulf Stream and it's related dynamics as in \todo{Not sure about this sentence} \citep{Greatbatch1991} and \citep{Zhang2007} among others.


\textcolor{gray}{
    The most common misrepresentation seems to be a separation from the coast to far south \td{NEEDS a citation}. This results in a cold bias in the north west corner of the Atlantic which has become known to modellers as the "blue spot of death" \citep{Gnanadesikan2007}. \td{Add figure?} \td{Lead into this a bit better - set up the link to the NRG and explain what it is? or leave this bit to later on} The Northern Recirculation Gyre (NRG) interacts with the Gulf Stream and it has been noted by \citep{Zhang2007} and \citep{Ezer2016b}\td{check this citation} that the representation of the Gulf Stream can have large effects on the energy modelled within the NRG\todo{interaction between the two rather than one to the other} \td{\& the subtropical gyre too?}. A lethargic NRG subsequently affects the \td{*blah* find out about this} and hence we can see that a poorly resolved Gulf Stream can have much wider implications throughout the model.} \todo{Remove this?}

\paragraph{Problems and Restrictions}
\ignore{ % TODO list
    \begin{itemize}
      \item Many people restricted to coarser resolution models due to the cost of high resolution models but need high resolution to get a more accurate Gulf Stream - it's expensive to be accurate
      \item We don't understand the smaller processes involved. \citep{Nikurashin2012a}
    \end{itemize}}
Higher resolution models have been shown to provide a more realistic Gulf Stream path \td{citations}. However this comes at a cost requiring more processing power and thus a more expensive model. 
It is understood \td{citations \citep{Nikurashin2012a} etc.} that this improvement is due to the higher resolution models being able to resolve small-scale processes which are lost in the coarser resolution counterparts. \td{This is discussed further in section blah.}
\citep{Ezer2016b} suggests that to replicate an accurate Gulf Stream separation, the model not only needed to resolve the Gulf stream but also the northern branches of NRG, the southward slope and shelf currents. \td{Could the bathymetry cause an unresolved northern branches of NRG?}.
\td{citation} It is also thought that the lack of detailed representation of the bathymetry in key areas play a large role in the unrealistic Gulf Stream path.
\td{citation} posed that it is the interaction with the bathymetry and the consequential small-scale processes which cause the Gulf Stream to split off from the coast at Cape Hatteras and direct it's path from there.
The bathymetry itself can be represented in many different ways and the vertical coordinate system chosen has a large impact on the interaction of the ocean dynamics and the ocean floor. \td{This is discussed in more detail in section *blah*.}
\textcolor{magenta}{Maybe add in here about scales - the different vertical levels and horizontal scales against the dimensions of the Gulf Stream?? *find citations* The Gulf Stream is ~100km wide and varies from a depth of 800-1200km. Coarser models of 1\degree resolve to 100km per cell while 0.25\degree models resolve ~25km per cell. These scales compared to the ~100km width of the Gulf Stream demonstrate that a higher resolution model should give a clearer representation. This is akin to trying to identify a pixelated image - the higher the resolution, the clearer the picture.}
\ignore{ \citep{Ezer2016b} revisited a discussion on different vertical coordinate systems and their effect on the path of the Gulf Stream, though noted that the tested z-coordinate system did not include any partial or shaved cells. \td{partial \& shaved cells - Mike's paper?}}

\paragraph{TODO List}
\begin{itemize}
	\item Include size and scale of GS (some details in Mike's paper) - also can mention about resolution - e.g. if it's clear it isn't resolved in 1\degree models or something!
	\item Maybe include this in talk about it's evolution
	\item Mention that it stopped etc.
	\item \citep{Scaife2011a} showed with their ocean only simulations that the SST bias off Newfoundland is significantly alleviated by increasing the horizontal resolution (from 1\degree to 0.25\degree) (End of section 4) - maybe include in Background too. The atmosphere only model provided mean bias and blocking error greatly alleviated at\textit both\textit resolutions.
\end{itemize}

\end{document}
