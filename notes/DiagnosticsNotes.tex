\documentclass[..\EOYR.tex]{subfiles}

\begin{document}

\section{Barotropic Vorticity Diagnostics}

Results so far - link into Background \& lead nicely into the next section.
\subsection{Barotropic vorticity diagnostics}
\begin{itemize}
  \item Results so far
  \item Discussion on results and possible implications
  \item Results on what we're looking at and why
  \item Maybe brief description of JEBAR etc. to explain it's use \& who else has used it to show what - e.g. why are we looking at it etc?
  \item Should probably include model set up
  \item Smoothing (laplacian?? see \citep{Bell1999}
  \item alpha-somethign model? Beth's thing/presentation!
  \item Expect to see similar results of break down of terms as \citep{Ezer2016b} - see fig 8?
\end{itemize}

Blhablha don't know how to start this section lalala!
Using the NEMO model, we are in the process of including some barotropic vorticity diagnostics to further investigate the processes effecting the path of the Gulf Stream. These barotropic vorticity diagnostics have been used by \citep{Bell1999},\citep{Gula???} and \td{other people} to examine \td{who knows what}. The momentum equations in the model are \td{momentum equations and terms}. Two diagnositcs will be calculated on each of the \td{XX} terms. \td{Better wording from Mike's paper: the curl can be taken of the terms in the vertical average/integral of the momentum equations.} \td{Integrate over depth.}
For each contribution to the momentum equations, we will first calculate the vertical integrals, $\bar{u}$ and $\bar{v}$, over the height of the water column from the depth, $-h$, to the sea surface level $\eta$,
\begin{equation}
	\bar{u} = \int_{-h}^{\eta} u dz ,\quad \bar{v} = \int_{-h}^{\eta} v dz 
\end {equation}

We will also calculate the vertical averages $\langle\bar{u}\rangle$ and $\langle\bar{v}\rangle$ over the height of the water column, 
\begin{equation}
	\langle\bar{u}\rangle = \frac{1}{\eta + h}\int_{-h}^{\eta}u dz ,\qquad \langle\bar{v}\rangle = \frac{1}{\eta + h}\int_{-h}^{\eta}v dz
\end{equation}
The first diagnostic will be obtained by taking \td{the vertical component of}\todo{check this} the curl.
\td{EQUATION}
\begin{equation}
	BVI(a) = \frac{\partial \bar{v}}{\partial x} - \frac{\partial \bar{u}}{\partial y}
\end{equation}
The second diagnostic is obtained by taking the vertical average of the momentum contribution over the depth \td{H} of th ocean and then taking \td{the vertical component of}\todo{check this} the curl. \todo{find a better wording of this!}
\begin{equation}
	BVA(a) = \frac{\partial \langle\bar{v}\rangle}{\partial x} - \frac{\partial \langle\bar{u}\rangle}{\partial y}
\end{equation}
	Using these diagnostics, we can calculate the barotropic \td{and lots of other things}\todo{find out what!!} streamfunction which will help to pinpoint \td{blahblahblah}.
\td{Go into each of the contributing factors - put the equation maybe? or at least pick them out from above - look at Mike's paper for help!!}

ln\_zps\=true : z-coordinate with partial step bathymetry


Prognostic Ocean dynamics equations summarised as:
\begin{equation}
    NXT = (VOR+KEG+ZAD)+HPG+SPG+LDF+ZDF
\end{equation}
    $(VOR+KEG+ZAD)$ are the coriolis \& advection terms. Then the prsesure gradient contributions (HPG - hydrostatic \& SPG - surface pressure), LDF - lateral diffusion \& ZDF - vertical diffusion
\citep{Madec2011} P102


\paragraph{Notes from \citep{Mertz1992}} Can interpret JEBAR two different ways. First as "the correction to the topographic vortex stretching term in the detph-averaged vorticity equation" - achieved by first depth averaging the momentum equations. Secondly by "relating it to the bottom torque, which is simply the curl of the horizontal force exerted by the bottom on the fluid" - for which we look at the equation of the evolution of the vorticity of the depth-integrated flow. -> this is why we do depth integral {\bf{and}} depth-average.

\paragraph{Random notes} As discussed in \td{section blah} the JEBAR term can be interpreted and arrived at from different directions, we apply the same approach to the diagnostics.\\
We apply the diagnostics calculated in \citep{Bell1999} to the NEMO ORCA025 model with GO6 forcing.\td{Model set up?}\\
The first diagnostic involves \td{Cross differentiating/taking the curl} of the vertical average of the contributions to the momentum. \\
This will allow us to interpret thes TODO

    \textcolor{magenta}{find out what the vertical velocity at the bottom of the model of the tracer grid is. In \citep{Bell1999} it's zero which is important...}

\paragraph{Notes from \citep{Bell1999}} In the first set of diagnostics (curl of the terms in the vertical integral of the momentum equations) the interaction with the bathymetry is related to the JEBAR.  In the second set of diagnostics (curl of the terms in the vertical average of the momentum equations) the interaction with the bathymetry is related to the bottom pressure torque.

\subsection{Bottom Pressure Torque}
\subsection{Vortex Stretching}

\end{document}
