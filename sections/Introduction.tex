\documentclass[..\EOYR.tex]{subfiles}

\begin{document}

\section{Introduction}

\todo{Include Gulf Stream dimensions - or maybe mention if talking about scaling at some point?}
Ocean eddies are the subject of many ongoing investigations into ocean dynamics. The topic comprises a broad range of smaller-scale dynamics and relationships at varying scales. From energy transfer to blah blah *citations*, researchers are still trying to get a better understanding of the internal dynamics of the ocean. *citations*. Our main focus will be on the area of the Gulf Stream and mainly the interaction and effect of the bathymetry on the separation and subsequent path of the Gulf Stream.
\todo{Haven't added citations}

\subsection{The impact of an accurately resolved Gulf Stream}
\begin{itemize}
  \item Impact on other weather events - e.g.  \citep{Scaife2011a}
  \item Effect on other ocean processes and circulation - "blue spot of death"    \todo{NW corner. Forntal path to west}
  \todo Global warming etc?
\end{itemize}
The Gulf Stream is part of the AMOC and is essential to transferring heat from the Gulf of Mexico at lower latitudes towards western and northern Europe. Fundamentally it gives north-western Europe its mild climate and thus has large impacts on the weather and climate as the heat is transferred into the atmosphere. This can have wide ranging effects on weather prediction – notably winter blocking \citep{Scaife2011a} and *other citation* \todo{I'm sure I read about cyclones somewhere - find citation}. Many lower resolution models render an unrealistic Gulf Stream, with the separation from the U.S. coast too far to the South, causing an inaccurate path from this point onwards. This has a knock-on effect as the corresponding sea surface temperatures (SSTs) are much higher/lower than observed in the surrounding areas \citep{Greatbatch2004}\todo{check this is the right citation}. The cold bias in the North West corner of the Atlantic resulting from a poorly placed Gulf Stream has become known as the "blue spot of death" \citep{Gnanadesikan2007} to modellers due to the inaccuracies. \todo{Lead into this a bit better - set up the link to the NRG and explain what it is? or leave this bit to later on} The Northern Recirculation Gyre (NRG) interacts with the Gulf Stream and it has been noted by \citep{Zhang2007} and \citep{Ezer2016b}\todo{check this citation} that the representation of the Gulf Stream can have large effects on the energy modelled within the NRG\todo{interaction between the two rather than one to the other} \todo{\& the subtropical gyre too?}. A lethargic NRG subsequently affects the *blah*\todo{find out about this} and hence we can see that a poorly resolved Gulf Stream can have much wider implications throughout the model.

\subsection{Problems and Restrictions}
\begin{itemize}
  \item Many people restricted to coarser resolution models due to the cost of high resolution models but need high resolution to get a more accurate Gulf Stream - it's expensive to be accurate
  \item We don't understand the smaller processes involved. \citep{Nikurashin2012a}
\end{itemize}
Higher resolution models have been shown to provide a more realistic Gulf Stream path *citations*, however this comes at a cost requiring more processing power and thus a higher price for the model. It is generally understood *citations \citep{Nikurashin2012a} etc.* that this improvement is due to the higher resolution models being able to resolve small-scale processes which are lost in the coarser resolution counterparts. \citep{Ezer2016b} suggests that to replicate an accurate Gulf Stream separation, the model not only needed to resolve the Gulf stream but also the northern branches of NRG, the southward slope and shelf currents. \todo{Could the bathymetry cause an unresolved northern branches of NRG?}. *citation* It is also thought that the lack of detailed representation of the bathymetry in key areas play a large role in the unrealistic Gulf Stream path. *citation* posed that it is the interaction with the bathymetry and the consequential small-scale processes which cause the Gulf Stream to split off from the coast at Cape Hatteras and direct it’s path from there. The bathymetry itself can be represented in many different ways and the vertical coordinate system chosen has a large impact on the interaction of the ocean dynamics and the ocean floor. \citep{Ezer2016b} revisited a discussion on different vertical coordinate systems and their effect on the path of the Gulf Stream, though noted that the tested z-coordinate system did not include any partial or shaved cells. \todo{partial \& shaved cells - Mike's paper?}

\para{TODO List}
\begin{itemize}
	\item Include size and scale of GS
	\item Maybe include this in talk about it's evolution
	\item Mention that it stopped etc.
\end{itemize}

\end{document}
