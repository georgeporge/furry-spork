\documentclass[..\EOYR.tex]{subfiles}

\begin{document}

\section{Background}
Tell a story up to where we are and what's missing.
\paragraph{Literature Review (take most from coursework).Cover the various angles including:}
\begin{itemize}
  \item Modelling \& vertical grid systems - include partial cells
  \item Turbulence and bathymetry - understanding the smaller processes (vorticity etc.) \citep{Tansley2001} \citep{Nikurashin2012a}
  \item Inclulde some phenomenology - e.g. bottom vortex stretching and bottom pressure torque. Maybe also the weird sandwhich diagram!
  \item Include context of climate modelling (e.g. Scaife et al, Ezer summary etc)
  \item context of what affects the Gulf Stream etc.
  \item Possible relationships/data sets to see the history of the Gulf Stream. (Palaeo?) \citep{Ezer2015}
  \item Cut down lit review - parts of it will be elsewhere!! (or should be! OR Remove them from elsewhere to keep them here!!)
    \item Bottom Pressure Torque -> JEBAR. JEBAR is part of bottom pressure torque (see Geoff's book \& \citep{Greatbatch1991}. The additional term corresponds to the torque of the vertically averaged pressure (\citep{Greatbatch1991})
\end{itemize}

\subsection{Gulf Stream Separation \& NAC}
\begin{itemize}
    \item Maybe lead in through the Gulf Stream intro?
    \item Frame it in ocean eddies etc and energy transfer?
\end{itemize}

\subsection{Subpolar Gyre \& subtropical gyre?}

\subsection{The Effects of topography}
\begin{itemize}
    \item Maybe introduct Streamfunction?
    \item Include Bottom Pressure Torque
    \item Include JEBAR
    \item Include Vortex Stretching etc. \citep{Zhang2007}
    \item \citep{Bell1999}
    \item Bottom Pressure Torque -> JEBAR. JEBAR is part of bottom pressure torque (see Geoff's book \& \citep{Greatbatch1991}. The additional term corresponds to the torque of the vertically averaged pressure (\citep{Greatbatch1991})
\end{itemize}

\subsubsection*{Bottom Pressure Torque}
\td{JUST NOTES}
\citep{Greatbatch1991} Fig 1 \& Fig 5 differences show the effect of bottom pressure torque missing from models which assume a flat bottom. They then have equations to demonstrate ((16) \& (17)). \citep{Greatbatch1991} say bpt forcing responsible for enhacnign max transport of the subtropical gyre which combined with significant bpt forcing to the SE of the Grand Banks is important for determining the max trasnport of the diagnosed gulf stram. \\
\td{See notes on Greatbatch paper in Papers doc for eqns - rewrite these in more modern notation - maybe consult Geoff's book!}

\subsubsection*{JEBAR}
\td{JUST NOTES}
\citep{Greatbatch1991} JEBAR accounts for part of the bottom pressure torque - Greatbatch has a good example of difference between solutions where JEBAR is 0 and other part is 0 to show the significance of JEBAR term.

Then JEBAR itself can be split into separate parst - see \citep{Greatbatch1991} and \citep{Gula2014}.\\

\td{See notes on Greatbatch paper in Papers doc for eqns - rewrite these in more modern notation - maybe consult Geoff's book!}

\subsubsection*{Bototm Vortex Stretching}

\subsection{Small Scale Processes}
\begin{itemize}
    \item Lots from lit review - and add some dissipation stuff.
\end{itemize}

\subsection{Energy Dissipation}
\begin{itemize}
    \item How the energy is changed from global eddies to smaller spin and/or lost. internal wave drag etc?
\end{itemize}

\subsection{Climate Models}
\begin{itemize}
    \item Talk about model resolution and improvements
    \item Mention Zanna paper???
    \item Different vertical coordinate systems
    \item Effects of the Gulf Stream in models etc. Scaife et al...
    \item Parameterisations and improvements etc. Work being done on the more modelly side!
\end{itemize}

\end{document}
