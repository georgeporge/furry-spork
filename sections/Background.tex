\documentclass[..\EOYR.tex]{subfiles}

\begin{document}

\section{Background}
Tell a story up to where we are and what's missing.
\paragraph{Literature Review (take most from coursework).Cover the various angles including:}
\begin{itemize}
  \item Turbulence and bathymetry - understanding the smaller processes (vorticity etc.) \citep{Tansley2001} \citep{Nikurashin2012a}
  \item Inclulde some phenomenology - e.g. bottom vortex stretching and bottom pressure torque. Maybe also the weird sandwhich diagram!
  \item Include context of climate modelling (e.g. Scaife et al, Ezer summary etc)
  \item context of what affects the Gulf Stream etc.
  \item Possible relationships/data sets to see the history of the Gulf Stream. (Palaeo?) \citep{Ezer2015}
  \item Cut down lit review - parts of it will be elsewhere!! (or should be! OR Remove them from elsewhere to keep them here!!)
  \item When introducing equations... maybe start with the basic equations then the vector invariant form of the momentum equation etc...????
  \item Modelling \& vertical grid systems - include partial cells
  \item Balance between acceleration by wind stress and decelleration by pressure force on bottom topography \citep{NaveiraGarabato2013} - he cites Vallis 2006 (Geoff's book!)??
\end{itemize}

\td{Do I need to introduce the background? Do I have it as a lit review? Or Split it into separate topics within my thingy???? WAAAAAAAAAAAH}

\subsection{Gulf Stream Separation \& NAC}
\begin{itemize}
    \item Maybe lead in through the Gulf Stream intro?
    \item Frame it in ocean eddies etc and energy transfer?
\end{itemize}



\subsection{Subpolar Gyre \& subtropical gyre?}

\subsection{The Effects of topography}
\begin{itemize}
    \item Maybe introduct Streamfunction?
    \item Include Bottom Pressure Torque
    \item Include JEBAR
    \item Include Vortex Stretching etc. \citep{Zhang2007}
    \item \citep{Bell1999}
    \item Bottom Pressure Torque to JEBAR. JEBAR is part of bottom pressure torque (see Geoff's book \& \citep{Greatbatch1991}). The additional term corresponds to the torque of the vertically averaged pressure (\citep{Greatbatch1991})
\end{itemize}

The horizontal components of the vector invariant form of the momentum equation are: (From Geoff's book P64)
\begin{equation}\label{VIMomentumU}
    \frac{\partial u}{\partial t} - (f+\zeta)v + w\frac{\partial u}{\partial z} = -\frac{1}{a cos \phi}(\frac{1}{\rho}\frac{\partial p}{\partial \lambda} + \frac{1}{2}\frac{\partial \mathbf{u}^2}{\partial \lambda})
\end{equation}
and
\begin{equation}\label{VIMomentumV}
    \frac{\partial v}{\partial t} - (f+\zeta)u + w\frac{\partial v}{\partial z} = -\frac{1}{a}(\frac{1}{\rho}\frac{\partial p}{\partial \phi} + \frac{1}{2}\frac{\partial \mathbf{u}^2}{\partial \phi})
\end{equation}

We start with the primitive horizontal momentum equations in spherical coordinates:
\td{Primitive: 3 approximations: (From Geoff's book)}
\begin{itemize}
    \item The hydrostatic approximation. In the vertical momentum equation the gravitational term is assumed to be balanced by the pressure gradient term, so that
        \begin{equation}
            \frac{\partial p}{\partial z} = -\rho g
        \end{equation}
        The advection of vertical veloicyt, the coriolis terms and the metric term $\frac{(u^2 + v^2)}{r}$ are all neglected.
    \item The shallow-fluid approximation. We write $r=a+z$ where the constant $a$ is the radius of hte Earth and $z$ increases in the radial direction. The coordinate $r$ is hten replaced by $a$ except where it is used as the differentiating argument. Thus, for example,
        \begin{equation}
            \frac{1}{r^2}\frac{\partial (r^2 w)}{\partial r} \to \frac{\partial w}{\partial z}
        \end{equation}
    \item The traditional approximation. Coriolis terms in the horizontal momentum equations involving the vertical velocity, and the still smaller metric terms $uw/r$ and $vw/r$, are neglected.
\end{itemize}

Then the governing equations become:
\begin{equation} \label{PrimMomUMB}
    \frac{\partial u}{\partial t}=-((\mathbf{u}.\nabla)u - \frac{uv\tan \phi }{a}) + fv - \frac{1}{\rho_0 a \cos \phi}\frac{\partial p}{\partial \lambda} + \frac{1}{\rho_0}\frac{\partial \tau_\lambda}{\partial z} + F_\lambda
\end{equation}
\begin{equation} \label{PrimMomVMB}
    \frac{\partial v}{\partial t}=-((\mathbf{u}.\nabla)v + \frac{u^2\tan \phi }{a}) - fu - \frac{1}{\rho_0 a}\frac{\partial p}{\partial \phi} + \frac{1}{\rho_0}\frac{\partial \tau_\phi}{\partial z} + F_\phi
\end{equation}

\td{Find a way to label the terms as nonlinear advection, coriolis acceleration, pressure gradient force, vertical diffusion of momentum, and horizontal diffusion of momentum.}\textcolor{magenta}{Maybe find an eqn with the 7 terms I'm using instead!?! have found the 7 terms in nemo - maybe work with this for now and come back to it later.}

\begin{equation} \label{PrimMomUGV}
    \frac{\partial u}{\partial t}=2\Omega\sin\phi v - (\mathbf{u}.\nabla)u + \frac{uv\tan \phi }{a}) - \frac{1}{\rho_0 a \cos \phi}\frac{\partial p}{\partial \lambda}
\end{equation}
\begin{equation} \label{PrimMomVGV}
    \frac{\partial v}{\partial t}=-2\Omega\sin\phi u - (\mathbf{u}.\nabla)v - \frac{u^2\tan \phi }{a}) - \frac{1}{\rho_0 a}\frac{\partial p}{\partial \phi}
\end{equation}


\td{Make sure to define $\phi$ as going from the equator!!! and $\lambda$ as the azimuth}



\subsubsection*{Bottom Pressure Torque}
\td{JUST NOTES}
\citep{Greatbatch1991} Fig 1 \& Fig 5 differences show the effect of bottom pressure torque missing from models which assume a flat bottom. They then have equations to demonstrate ((16) \& (17)). \citep{Greatbatch1991} say bpt forcing responsible for enhacnign max transport of the subtropical gyre which combined with significant bpt forcing to the SE of the Grand Banks is important for determining the max trasnport of the diagnosed gulf stram. \\
\td{See notes on Greatbatch paper in Papers doc for eqns - rewrite these in more modern notation - maybe consult Geoff's book!}
\citep{Yeager2015} "The relationship between JEBAR and BPT has been clarified by, among others, Mertz and Wright (1992), Greatbatch et al. (1991), and Bell (1999); the former arises in the (potential) vorticity equation of the vertically averaged horizontal flow, whereas the latter arises in the vorticity equation of the vertically integrated horizontal flow (see appendixA). JEBAR represents the component of BPT associated with the baroclinic (buoyancy de- pendent) part of the pressure gradient, and therefore it vanishes in the absence of stratification (Mertz andWright 1992; Salmon 1998). BPT can be nonzero regardless of stratification because it represents the projection of hori- zontal geostrophic bottom flow normal to isobaths. With the condition of no normal flowat the ocean bottom,BPT can be understood as a geostrophic bottom vortex stretching" (See eq (1) in paper).

We start with the \td{blah blah} of the momentum equation,
\begin{equation}\label{momentum}
\end{equation}

\subsubsection*{JEBAR}
\td{JUST NOTES}
\citep{Greatbatch1991} JEBAR accounts for part of the bottom pressure torque - Greatbatch has a good example of difference between solutions where JEBAR is 0 and other part is 0 to show the significance of JEBAR term.

Then JEBAR itself can be split into separate parst - see \citep{Greatbatch1991} and \citep{Gula2014}.\\

\td{See notes on Greatbatch paper in Papers doc for eqns - rewrite these in more modern notation - maybe consult Geoff's book!}


The Joine Effect of Baroclinicity And Relief (JEBAR) was first introducted to account for the effects of topography and baroclinicity in ocean dynamics. It is these effects which balance with the wind stress. This is terrible writing oh my god what areyou doing. just write something half way decent. In fact just write anything. Find an equation to write down to explain what JEBAR is.

\subsubsection*{Bototm Vortex Stretching}

\subsection{Small Scale Processes}
\begin{itemize}
    \item Lots from lit review - and add some dissipation stuff.
\end{itemize}

\subsection{Energy Dissipation}
\begin{itemize}
    \item How the energy is changed from global eddies to smaller spin and/or lost. internal wave drag etc?
\end{itemize}

\subsection{Climate Models}
\begin{itemize}
    \item Talk about model resolution and improvements
    \item Mention Zanna paper???
    \item Different vertical coordinate systems
    \item Effects of the Gulf Stream in models etc. Scaife et al...
    \item Parameterisations and improvements etc. Work being done on the more modelly side!
\end{itemize}

\end{document}
